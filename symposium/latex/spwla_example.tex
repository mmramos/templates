\documentclass[10pt,twocolumn,twoside]{article}
\usepackage{graphicx,xcolor}
\usepackage{wrapfig}
\usepackage{amsmath}
\usepackage{amssymb}
\usepackage{bm, bbm} % bold math
\usepackage{subfigure}
\usepackage{natbib}
\usepackage{enumitem}
\usepackage{float}
%\usepackage{mathptmx}
%\usepackage[labelfont=bf]{caption}
%\usepackage{epsfig,bm}
%\usepackage{amsfonts}
%\usepackage{lipsum}
%\usepackage{titlesec}
%\usepackage{titling}
\usepackage{spwla}
%\usepackage{showframe}
%\bibpunct{(}{)}{;}{a}{ }{,}
%\usepackage[labelfont={bf},justification=RaggedRight,textfont=normal]{caption}
\usepackage{caption}

%\captionsetup{labelfont=bf,textfont=normal,justification=justified,labelsep=endash,aboveskip=2pt}
\captionsetup{labelfont=bf,textfont=normal,justification=justified,aboveskip=2pt}
\PassOptionsToPackage{hyphens}{url}\usepackage{hyperref}
\usepackage[export]{adjustbox}
\usepackage[capitalize, nameinlink]{cleveref}
\crefdefaultlabelformat{#2\textbf{#1}#3} % <-- Only #1 in \textbf
\crefname{figure}{\textbf{Figure}}{\textbf{Figures}}
\Crefname{figure}{\textbf{Figure}}{\textbf{Figures}}
\crefname{table}{\textbf{Table}}{\textbf{Tables}}
\Crefname{table}{\textbf{Table}}{\textbf{Tables}}
%\newcommand{\rs}[1]{\mathstrut\mbox{\scriptsize\rm #1}}
%\newcommand{\rr}[1]{\mbox{\rm #1}}
%\newcommand*\mean[1]{\bar{#1}}
%\newcommand*\tran[1]{\tilde{#1}}
%use below 4 rows to avoid hyphenation
%\tolerance=1
%\emergencystretch=\maxdimen
%\hyphenpenalty=10000
%\hbadness=10000
\def\headername{{SPWLA 65$^\textbf{th}$ Annual Logging Symposium, May 18--22, 2024}}

\setcounter{page}{1}

\renewcommand{\thesubfigure}{\thefigure\alph{subfigure}}
\makeatletter
   \renewcommand{\p@subfigure}{}
   \renewcommand{\@thesubfigure}{(\alph{subfigure})\hskip\subfiglabelskip}
\makeatother

\renewcommand\sectionmark[1]{%
	\markright{\thesection\ #1}}

%\setcounter{page}{11}

\newcommand{\Erf}{\operatorname{Erf}}
\newcommand{\var}{\operatorname{var}}
\newcommand{\cov}{\operatorname{cov}}
\def\frech{Fr\'{e}chet\ }
%
% for vector symbols; write argument in math and bold
% can be used inside or outside mathmode
%
\newcommand{\vc}[1]{\ensuremath{\mathbf{#1}}}
%
% transpose for matrices (math mode only)
%
\newcommand{\trp}{^{\text{\scriptsize T}} }
\newcommand{\itp}{^{-\text{\scriptsize T}} }
\newcommand{\inv}{^{-1} }
%
%  Super- or subscript in roman
%  in math mode or parmode
%
\newcommand{\sbr}[1]{\ensuremath{_{\mathrm{#1}}}}
\newcommand{\spr}[1]{\ensuremath{^{\mathrm{#1}}}}
%
%   Bibliography style

\bibliographystyle{spwla}


%****** REFERENCES *************************************************************
\def\thebibliography#1{\section{References}
\list
{\arabic{enumi}.}
{\settowidth\labelwidth{#1.}\leftmargin\labelwidth
\advance\leftmargin 0pt \labelsep 5pt \labelwidth 20pt \itemindent -10pt
\usecounter{enumi}}     \def\newblock{\hskip .11em plus .33em minus -.07em}
\sloppy \sfcode`\.=1000\relax
}\let\endthebibliography=\endlist

%
%   Title
%
\title{SPWLA Symposium general instructions for paper manuscript preparation}
%
%   Author(s)
%
\author{Mario Martins Ramos, Fernando Vizeu, Rodrigo Dutra, José Augusto Vitorino Dias, João Vitor Alves Estrella, Jordan Jusbig Salas Cuno, Ana Carla dos Santos Pinheiro, Antonio Fernando Menezes Freire, Wagner Moreira Lupinacci}

% Headers
\headsep = 0.5 in
\footskip = 0.15 in
\renewcommand{\headrulewidth}{0pt}
\renewcommand{\footrulewidth}{0pt}
\fancyhead[RO]{\bf{\headername}}
\fancyhead[LE]{\bf{\headername}}
\fancyfoot[C]{\color{gray}\thepage}

\begin{document}
\thispagestyle{empty}
%\small                   % 9pt is default size, but there is no 9pt article.
\twocolumn[\maketitle]   % set the title as one-column, rest two-column.
%\maketitle

\noindent \parbox[t]{75mm}{ \scriptsize \rmfamily \noindent
	
	Copyright 2024, held jointly by the Society of Petrophysicists and Well Log Analysts (SPWLA) and the submitting authors.
	
	This paper was prepared for the SPWLA 65$^\text{th}$ Annual Logging
        Symposium held in Rio de Janeiro, RJ, Brazil, May 18--22, 2024.
}
\textcolor[rgb]{0.00,0.00,0.00}{\rule{75mm}{0.3mm}}

\fancypagestyle{firststyle}
{
	\fancyhf{}
	\fancyhead[R]{\bf{\headername}}
	\fancyfoot[C]{\footnotesize \thepage}
}

\thispagestyle{firststyle}

\section{abstract}
The meteoric rise of the Python programming language did not happen in isolation. Alongside this development, there was a notable amplification in the processing capacity of contemporary computers. Together, these technological advances made it possible for computational operations, which were previously considered expensive and time-consuming, to become more accessible and implementable. This is especially true in the context of academic centers, where research and technological development find fertile ground. Within the vast field of research, petrophysical problems have reaped the benefits of these innovations. With expanded processing capabilities and the versatility of Python, researchers now have the ability to work with massive data sets, especially when it comes to oil wells. Operations ranging from simple data extraction from files in the LAS and DLIS formats, to tasks demanding intense computational power, like machine learning and inversion techniques, have become more efficient and achievable. However, despite these promising advances, one challenge persists. Many of the routines and algorithms that are meticulously developed during academic research are set aside after these activities conclude. These algorithms, rich in potential application, are forgotten, and if there's an interest in reusing them later on, there is often a need to almost redevelop from scratch. It is in this context that the relevance of this work arises. We aim to present practical solutions to this challenge, and one of the pillars of this solution is the software APPy. This software aims to bridge the gap between researchers, both those with deep knowledge in programming and those unfamiliar with the field. APPy offers a platform where the production, reuse, and optimization of algorithms become more integrated and efficient processes, eliminating the barrier that often prevents the reuse of valuable solutions. In this way, we seek a more connected academia, where knowledge is shared and maximized for the benefit of research and practical application.


\section{introduction}
Python, over the years, has seen phenomenal growth, and it has become one of the most popular programming languages worldwide. Initially developed in the late 1980s by Guido van Rossum, Python's simplicity, readability, and versatility have made it the first choice for many developers and industries \citep{pysftw2021}. Its growth can be attributed to a combination of factors. One significant reason is its wide application in emerging fields, such as data science, artificial intelligence, machine learning, and web development.

Several studies and surveys, including those from Stack Overflow, have highlighted Python's soaring popularity. Stack Overflow, a major platform for developers to learn and share their knowledge, has observed an increasing number of questions related to Python over the years \citep{stackoverflow2017}. This is indicative of a growing community of Python developers and enthusiasts. Additionally, the language's extensive libraries and frameworks, coupled with a supportive community, have made Python an attractive choice for both beginners and experienced developers (\cref{fig:languages}).

\begin{figure}[H]
	\centering
	\adjincludegraphics[width=3.125in, trim={{.0
			\width} {.0\width} {.0\width} {0.0\width}},clip]%
	{figs/fig_1.png}
	%{Overview_examples_crop.png}
	\caption{Most popular languages survey \citep{stackoverflow2019,stackoverflow2022}.}
	\label{fig:languages}
\end{figure}

Also, based on the most recent data from Stack Overflow's annual developer survey, JavaScript maintains its status as one of the most popular and widely used programming languages in the world (\cref{fig:languages}). For several years running, an overwhelming majority of developers have reported using JavaScript, with its dominance evident across both front-end and back-end development domains \citep{jsstate2021}. The data suggests that the language's adaptability, coupled with the continual emergence of modern libraries and frameworks, has solidified its importance in the ever-evolving landscape of web technologies. Additionally, the active and expansive community support evident on platforms like Stack Overflow indicates that JavaScript's relevance is not merely a reflection of its historical significance but also a testament to its current vitality and potential for future innovation.

Considering database language, SQL's enduring relevance stems from its ubiquity across most relational database management systems, its standardization by ANSI, the optimized performance of relational databases, its maturity from decades of use, and its persistent demand in various tech job roles \citep{winand2012}. However, its prominence might be challenged in the future by the rise of NoSQL databases, potential advancements in newer query languages, and integrated platforms that provide query solutions without the direct need for SQL \citep{avantika2022}. Despite these challenges, as of 2023 to 2024, SQL's foundational position in data management suggests it will likely remain significant for years to come \citep{stackoverflow2022}.

The rapid growth of popularity of programing languages, results in one last question to answer: Is still valid to develop softwares, or everyone should learn some programing languages? While the production of software remains crucial in our technologically driven world \citep{usbureau2023}, with a growing demand seen in sectors like machine learning, IoT, and augmented reality, the need for individuals to learn programming is more nuanced. A basic understanding of programming concepts is beneficial for many, akin to how understanding arithmetic doesn't make everyone a mathematician. As digital transformation becomes more pervasive, underscored by reports like the World Economic Forum's "The Future of Jobs Report 2020" \citep{weforum2020} it's clear that a foundation in digital literacy is becoming increasingly important. However, this doesn't necessarily translate to proficiency in specific programming languages for everyone. Instead, understanding the logic and problem-solving approach inherent to programming may be more vital.

Consequently, despite the substantial proliferation of programming languages, there remains a valid need for the development of software applications to cater to the increasing demand for emergent technologies. Furthermore, the employment of the aforementioned technologies (Python, SQL, and Javascript) appears to be of significant relevance in the development of such software applications. 

Besides the global trend that drives software development in multiple application fields, the specific scenario of petrophysical software outlines a particularly distinctive niche within the technological spectrum. This software area is often highlighted, especially in the academic context, due to a series of challenges and opportunities that fuel innovation. Firstly, the cost associated with the licenses of these softwares are often high, making it frequently unfeasible given the budget constraints of many academic and research projects. Additionally, the operational opacity and the lack of transparent documentation regarding the functionalities and methods implemented in these software programs limit adaptability and customization. This limitation becomes particularly significant when considering the implementation of new methodologies and approaches, a challenge for both experienced developers and professionals who lack extensive programming experience. Thus, there's a growing demand for the development of more accessible, transparent, and adaptable solutions in the realm of petrophysical software. Therefore, this ``environment'' was conducive to the emergence of the petrophysical evaluation software APPy.


\section{Methodological premises of the APPy software}
The foundational principles guiding the development of the APPy software were twofold. First and foremost, it sought to address the primary requirements for software tailored to the academic needs pertaining to petrophysical evaluation within university environments. Secondly, it was imperative for APPy to establish itself as both competitive and dependable in the broader software landscape. Subsequent sections will delve deeper into these specific requirements and the rationale behind their prioritization.

\subsection{Academic necessities and development environment}
Petrophysics is commonly intertwined with geophysics, engineering, and geology. A significant portion of students in these fields have encountered, or will encounter, Python at some juncture in their academic journey \citep{guo2014}. It's thus anticipated that many students will incorporate Python into their research. Even those who don't work with Python directly often possess a foundational understanding of the language, enough to utilize code written by others. This familiarity has made Python a unifying language among students, fostering collaboration and innovation.

Given this context, the foundational design principle for the APPy software is its identity as a 'Python-centric' tool. This entails that the majority of its functionalities are linked to the Python language, ensuring that most students can effortlessly interact with and enhance the software to meet their specific academic requirements or those of their research group.

Adopting Python in academic settings addresses three primary challenges posed by other software:

\begin{itemize}
	\item Cost-efficiency: Python is free.
	\item Transparency: As an open-source, interpreted language, Python's workings are accessible and understandable.
	\item Modularity: Python boasts a plethora of libraries, including those tailored for geosciences and petrophysics.
\end{itemize}

While these advantages are particularly appreciated by students familiar with Python, they may pose barriers for those who don't regularly engage with the language.

\subsection{Competitiveness in the broad software landscape}

To ensure that the software are able to acquire some competitiveness in the present day, even in its development stage, its imprecidible that it follows two principal development guidelines: microservices and incremental development.

\section{type style and size}
Type style (font). A proportionally spaced, serif font, Times or Times Roman, is preferred. If this is not possible, use the closest approximation available. (This is Times New Roman, 10-point.)

Type size and line spacing. The main body of the text is single-spaced in 10-point type; line spacing is 12-point (resulting in 6 lines per inch).

\section{headings}

Title. The title should be centered and typed in all capital letters, bold, 14-point, single spaced.

Headings. These should be on a separate line, left justified, easily distinguished from each other and clearly separated from the main body of the text. Bold upper case 10-point, 12-pitch.

First-Level Subheadings. Subheadings are subordinate to headings. The first subheading should be on a separate line above the text. Bold, lowercase, 10-point, 12-pitch line spacing.

Second-Level Subheadings. Any additional subheadings should be on same lines as text. Italics, lowercase, 10-point, 12-pitch line spacing.

\section{citations}

Figures. All figures (graphs, line drawings, photographs, etc.) should be cited in the body of the paper and should be numbered sequentially, see~\cref{fig:logo}. The figures in the text should be labeled as “\textbf{Figure 1}” in bold, 10-point, 12-pitch line spacing.

Tables. All tables should be cited in the body of the paper. Number the tables sequentially as they appear in the paper and include a caption consisting of the table number and a brief description. Tables should be inserted as part of the text as close to its first reference as possible. The tables should be labeled as “\textbf{Table 1}” in bold, 10-point, 12-pitch line spacing.

\begin{figure}
	\centering
	\adjincludegraphics[width=3.125in, trim={{.0
			\width} {.0\width} {.0\width} {0.0\width}},clip]%
	{SPWLA-2023-Logo.png}
	%{Overview_examples_crop.png}
	\caption{SPWLA 2023 logo.}
	\label{fig:logo}
\end{figure}


Equations. Equations should be numbered consecutively beginning with (1) throughout the entire paper. The number should be enclosed in parentheses and set flush right in the column on the same line as the equation. It is this number that should be used when referring to equations within the text. Equations should be referenced within the text as “Eq. (x).”

References. Cite references by author’s last name and year. For two authors write both names, for more than two authors use the format “author1 et al.” Cite all references and include complete information for each citation in "References" section.

\section{references}

Order. References~\citep{Archie1942} should be listed ALPHABETICALLY by the author’s last name. In case of multiple listings by the same author(s), references are listed by date (earliest first) and then alphabetically. Do not use abbreviations. Form and punctuation shown in the examples below should be observed.

Format. Author's last name, followed by their initials, year of publication, title of paper or article (capitalize only the first word of the title and any proper nouns), full name of journal (do not underline or use italics), specific volume number shown as Vol. \#, show beginning and ending page numbers. 10-point, 12-pitch line spacing.

For books (as applicable): edition, volume, series, chapter, pages, full name and location of publisher.

For journals or other periodicals (as applicable): name of publication, volume, issue, page numbers, DOI (if available).

For conference papers (as applicable): name, location and date(s) of conference, type of presentation, paper number, DOI (if available). 

For websites: Author or Site Name. YEAR. Title (page or article). Site name, date posted, web address (accessed date). 

A free resource to find the DOI of a paper is https://www.crossref.org/guestquery/ 

\section{“about the author” section}
Each author should include a biographical sketch. These should be in the same order as the author listing at the beginning of the paper. Quality photos are encouraged, provided the overall length of the article does not exceed 30 pages.

\section{artwork preparation}
Identification and captioning. All figures should be identified by a number (“\textbf{Figure 1}”) in bold 10-point font. Number all figures in sequential order and follow with a brief but descriptive caption. 

Location. Illustrations should be placed within the body of text if at all possible, but it is recognized that in certain cases it will be necessary to place them at the end of the paper.

Multiple figures per page. Authors are responsible for arranging multiple figures on each page within the columns where appropriate, or if it is not possible, then at the end of the paper. All lettering must be legible.

Figures resolution. Authors are encouraged to produce high resolution figures before they are inserted in the paper. We recommend illustrations at a resolution of at least 300 dots per inch (dpi) but 600 is preferred. Several petrophysical and scientific software packages already produce high resolution figures. However, a graphics-editing application may be helpful for preparing illustrations. 

\section{use of color}
Use of color in digital Transactions Volume. The Transactions Volume will include the pdf files, with figures exactly as submitted by the author. Figures may be in color or in black and white at the author’s discretion. However, the author should perform a test print in black and white to ensure legibility, as many readers will print the paper in black and white. 

\section{commercial advertising}
Logos. Commercial (Company) logos are not to be used in manuscripts.  Manuscripts will be rejected if company logos are present. 

Proprietary names. Generic names should be used instead of proprietary tool or program names. However, there might be cases where a trademark is unavoidable, it can only be used once and identified as such.

Please remember that the goal of the Symposium is to promote technical advancement and exchange. Accordingly, commercially oriented papers are strongly discouraged.

\section{permission/copyright}
The author is responsible for obtaining permission to use previously published material. A signed letter of permission from the copyright holder must be submitted to the SPWLA Business Office by mail to SPWLA, 8866 Gulf Freeway, Suite 320, Houston, TX 77017). Alternatively, a scan of the executed document may be emailed to stephanie@spwla.org 

The copyright document can be found at http://www.spwlaworld.com/ 

\section{changes and additions}
Please, re-submit a paper to Stephanie Turner at stephanie@spwla.org if changes
are needed after the original submission. No changes or additions will be
accepted after the April 24, 2023 deadline.

\section{summary}
We continue our commitment to provide a high-quality Transactions Volume. Please
address any suggestions for improvement to Stephanie Turner at the SPWLA
Business Office (stephanie@spwla.org) or Iulian Hulea, SPWLA Vice President Technology 
vp-technology@spwla.org

Thank you for your cooperation.

\section{appendix: petrophysical model}

This appendix describes the petrophysical model assumed in the paper and provides a detailed derivation of the associated equations.

\section{acknowledgments}
The authors would like to acknowledge ...

\section{nomenclature}
\makebox[1.5cm][l]{$bhr$}  Borehole radius, inch\\
\makebox[1.5cm][l]{$dtm$}  Mud slowness, $\mu$s/ft\\
\makebox[1.5cm][l]{$rhom$}  Mud density, kg/m$^3$

\bibliography{spwla_ref}

%\newpage

\section{about the authors}
\setlength\intextsep{0pt}
\begin{wrapfigure}{l}{0.33\linewidth}
	\includegraphics[width=1.0\linewidth]{SPWLA-2023-Logo.png}
	\vspace{0.03in}
\end{wrapfigure}
\textbf{SPWLA} is a Non-Profit corporation founded in 1959. Dedicated to the advancement of petrophysics focusing on log and core measurements, formation evaluation techniques, hydrocarbon, mineral and water resources.

\begin{wrapfigure}{l}{0.33\linewidth}
	\includegraphics[width=1.0\linewidth]{SPWLA-2023-Logo.png}
	\vspace{0.03in}
\end{wrapfigure}
\textbf{Author 2} is a Non-Profit corporation founded in 1959. Dedicated to the advancement of petrophysics focusing on log and core measurements, formation evaluation techniques, hydrocarbon, mineral and water resources. is a Non-Profit corporation founded in 1959. Dedicated to the advancement of petrophysics focusing on log and core measurements, formation evaluation techniques, hydrocarbon, mineral and water resources.

\end{document}
